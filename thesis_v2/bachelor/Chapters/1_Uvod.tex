\chapter{Úvod}
\label{sec:Introduction}
V dnešní době, kdy jsme obklopeni spoustou moderních technologií, existující programovací jazyky se stále vyvíjejí kupředu a nové postupně vznikají, existuje nespočet nástrojů, rozšíření, vývojových prostředí, které práci a vývoj na těchto jazycích dokážou v mnoha ohledech usnadnit. Jazyk MonkeyC, jenž bude středobodem této práce, zatím nedisponuje tak široce obsáhlou komunitou, jako mají např. v dněšní době velmi populární Javascript, Python, C Sharp, Ruby, atd...\\

Typickým příkladem společnosti, která se zabývá právě vývojem softwarů pro programátory či vývojáře, je česká JetBrains s.r.o. Ti mají na kontě již mnoho produktů usnadňujících programování, např. ReSharper (.NET), IntelliJ IDEA (Java, Groovy, atd...) či PyCharm (Python). Ovšem na podporu Monkey C zatím žádné softwarové řešení v JetBrains do světa nevypustili. \cite{jetbrains} \\

Na oficiálním webu \cite{marketplace} sice najdeme několik nástrojů, které s vývojem v Monkey C pomáhají, my ale budeme chtít, aby naše aplikace poskytovala co největší podporu uživateli a hlavně, aby byla postavené na našem vlastním řešení.\\

V následujících kapitolách se tedy budeme věnovat tvorbě a vývoji rozšíření pro vývojové prostředí Visual Studio Code, které poskytne plnou podporu při vývoji aplikací. Jako první si představíme jazyk Monkey C, jak se s ním pracuje, jaké aplikace s ním můžeme vyvíjet atd... Pro tvorbu aplikace využijeme jazyk Typescript, dále pak nástroj ANTLR, který se schopný vygenerovat předladač jazyka z jeho popisu obsaženého v bezkontextové gramatice. Další část práce bude věnováná tématům, jako jsou parsování kódu, vytvoření syntaktického a sémantického stromu, našeptávání kódu, atd...
%Jedná se o dynamicky postavený jazyk, stejně jako např. python, R, atd… Jazyk se používá k vývoji aplikací pro Garmin zařízení.\\ V další kapitolách budou detailně popsány klíčové komponenty k vytvoření finální aplikace, např. gramatika pro popis jazyka, ANTLR parser, atd... 
\endinput
% Nejprve uvedeme tridu dokumentu s volbami
\documentclass[czech,bachelor]{diploma}
% Dalsi doplnujici baliky maker
\usepackage[autostyle=true,czech=quotes]{csquotes} % korektni sazba uvozovek, podpora pro balik biblatex
\usepackage[backend=biber, style=iso-numeric, alldates=iso]{biblatex} % bibliografie
\usepackage{dcolumn} % sloupce tabulky s ciselnymi hodnotami
\usepackage{subfig} % makra pro "podobrazky" a "podtabulky"
\usepackage[cpp]{diplomalst}
\usepackage[a-1b]{pdfx}
% Zadame pozadovane vstupy pro generovani titulnich stran.
\ThesisAuthor{Jakub Pšenčík}

\ThesisSupervisor{Ing. Jan Janoušek}

\CzechThesisTitle{Podpora jazyka Monkey C v prostředí VS Code}

\EnglishThesisTitle{Monkey C Language Support in VS Code}

\SubmissionYear{2021}

% Pokud nechceme nikomu dekovat makro zapoznamkujeme.
\Acknowledgement{Rád bych na tomto místě poděkoval svému vedoucímu práce Ing. Janu Janouškovi za odborné a metodické vedení, ochotu a pomoc v průběhu vypracovávání práce.}

\CzechAbstract{V této bakalářské práci se budu zabývat vývojem rozšíření pro Visual Studio Code, jenž bude poskytovat podporu pro jazyk Monkey C. V teoretické části dojde k představení prostředí Visual Studio Code a nahlédnutí do problematiky vývoje rozšíření v tomto prostředí. Dále bude představen také jazyk Monkey C. V praktické části práce bude popsán nástroj ANTLR, který je schopen generovat vlastní překladač jazyka pomocí bezkontextové gramatiky, parsováním kódu a jeho syntaktickou analýzou. Dále bude v praktické části rozebrán návrh a implementace rozšíření, společně s popisem jednotlivých jeho částí. Závěrem bude výsledné rozšíření testováno a výsledky zhodnoceny.}

\CzechKeywords{bakalářská práce, rozšíření, Monkey C, Typescript, parser}

\EnglishAbstract{In this bachelor thesis I will deal with the development of extension for Visual Sstudio Code, which will provide full support for Monkey C language. In thoretical part will be introduced Visual Studio Code environment and insight into the development of extensions in this environment.The practical part of the thesis will describe the ANTLR tool, which is able to generate its own language compiler using context-free grammar, code parsing and its syntactic analysis. Furthermore, the practical part will discuss the design and implementation of the extension, along with a description of its individual parts. Finally, the resulting extension will be tested and the results evaluated.}

\EnglishKeywords{bachelor thesis, extension, Monkey C, Typescript, parser}

\AddAcronym{ANTLR}{ANother Tool for Language Recognition}
\AddAcronym{AST}{Abstract syntax tree}
\AddAcronym{VS Code}{Visual Studio Code}
\AddAcronym{API}{Application Programming Interface}

% odkaz na literaturu
\addbibresource{Citace.bib}

% Novy druh tabulkoveho sloupce, ve kterem jsou cisla zarovnana podle desetinne carky
\newcolumntype{d}[1]{D{,}{,}{#1}}

% Zacatek dokumentu
\begin{document}

% Nechame vysazet titulni strany.
\MakeTitlePages

% Jsou v praci obrazky? Pokud ano vysazime jejich seznam a odstrankujeme.
% Pokud ne smazeme nasledujici dve makra.
\listoffigures
\clearpage

% Jsou v praci tabulky? Pokud ano vysazime jejich seznam a odstrankujeme.
% Pokud ne smazeme nasledujici dve makra.
\listoftables
\clearpage

% A nasleduje text zaverecne prace.
\chapter{Úvod}
\label{sec:Introduction}
V dnešní době, kdy jsme obklopeni spoustou moderních technologií, existující programovací jazyky se stále vyvíjejí kupředu a nové postupně vznikají, existuje nespočet nástrojů, rozšíření, vývojových prostředí, které práci a vývoj v těchto jazycích dokážou v mnoha ohledech usnadnit. Jazyk Monkey C, jenž bude středobodem této práce, zatím nedisponuje tak široce obsáhlou komunitou, jako mají např. v dněšní době velmi populární Javascript, Python, C Sharp, Ruby, atd...\\

Typickým příkladem společnosti, která se zabývá právě vývojem softwarů pro programátory či vývojáře, je česká JetBrains s.r.o. Ti mají na kontě již mnoho produktů usnadňujících programování, např. ReSharper (.NET), IntelliJ IDEA (Java, Groovy, atd...) či PyCharm (Python). Ovšem na podporu Monkey C zatím žádné softwarové řešení v JetBrains do světa nevpustili. \cite{jetbrains} \\

Na oficiálním webu, kde Microsoft nabízí nejrůznější rozšíření  \cite{marketplace}, je sice možné nalézt několik nástrojů, které s vývojem v Monkey C pomáhají. Cílem této práce je však tvorba rozšíření, které poskytne uživateli co největší podporu a hlavně, aby byla postavena na vlastním řešení.
\\
Následujících kapitoly tedy budou věnovány tvorbě a vývoji rozšíření pro vývojové prostředí Visual Studio Code, které poskytne plnou podporu při vývoji aplikací v Monkey C. V kapitole 2 dojde k představení samotného jazyka Monkey C a operačního systému Connect IQ, na kterém Monkey C běží. V kapitole 3 jsou popsány veškeré nástroje a komponenty, které budou při tvorbě rozšíření použity. Pro tvorbu rozšíření bude využit jazyk Typescript, dále pak nástroj ANTLR, který je schopen vygenerovat překladač jazyka. Toho je schopen dosáhnout za pomoci jeho popisu obsaženého v bezkontextové gramatice, jenž byla pro tuto práci poskytnuta, a tudíž její vytváření nebylo součástí práce. Kapitola 4 se popisuje analýzu kódu jak z pohledu syntaxe, tak sémantiky. V kapitole 5 se práce zabývá již samotným návrhem a implementací rozšíření. Objeví se zde témata, jako parsování kódu, automatické doplňování a našeptávání kódu, atd... Kapitola 6 je poté věnována testování rozšíření.
 
\endinput
\chapter{Jazyk Monkey C} \label{2Chapter}
Monkey C \cite{monkeyc_2021} je objektově orientovaný jazyk. Byl navržen americkou společností Garmin Ltd. \cite{GARMIN_OFFICIAL} a jeho hlavním účelem je snadný vývoj Connect IQ aplikací pro nositelná zařízení (nejvhodnějším příkladem jsou zde chytré hodinky). Jedná se o dynamický programovací jazyk, podobně jako Java, PHP či Ruby. Z těchto uvedených jazyků také Monkey C vychází. Cílem Monkey C je zjednodušit proces vytváření samotné aplikace a umožnit tak vývojářům více se soustředit na zákazníka a méně na omezení zdrojů. Využívá tzv. "reference counting" k automatickému čištění paměti, což vývojáře osvobozuje od manuální správy paměti (např. jako v jazyce C/C++).
\\
Všechny aplikace, naprogramované v Monkey C, běží na operačním systému (vývojářské platformě) Connect IQ. \cite{Garmin_Connect_IQ} Connect IQ umožňuje jak jednotlivcům, tak velkým společnostem vyvíjet nespočet různých aplikací a rozšíření pro Garmin zařízení. Můžeme jej tedy zařadit mezi takové platformy, jako jsou Android či iOS. S pomocí Connect IQ je tedy možné vyvíjet např.:

\begin{enumerate}
\item \textbf{aplikace} - jedná se o plně funkční aplikace běžící na hodinkách. Může se jednat např. o hudební aplikaci, která sychronizuje obsah s mobilní aplikací na zařízení uživate (např. Spotify, iTunes, atd...)
\item \textbf{ciferníky} (Watch Faces) - ciferník si lze představit, jako domovskou obrazovku na telefonu, které uživateli dokáže zobrazit celou řadu informací (záleží samozřejmě na preferencích konkrétního jedince). Ciferník je na hodinkách aktualizován každých 60 vteřin a běží nepřetržitě v režimu nízké spotřeby.
\item \textbf{widgety} - opět se jedná o komponentu, která je běžně využívána na mobilních zařízeních. Widget dokáže, za pomocí dat z hodinek nebo připojeného telefonu, zobrazovat nejrůznější informace od počasí, přes puls uživatele až po notifikace příchozích hovorů. 
\end{enumerate}

Na obrázku \ref{img:monkeyC_Fragment} je možnost vidět jednoduchý fragment kódu v Monkey C. Z obrázku je patrné, že Monkey C, stejně jako většina dnešních programovacích jazyků, podporuje operace, jako dědičnost, určení rozsahu jmenného prostoru pomocí klíčového slova using, a mnoho dalších. "Stejně jako je tomu v programovacím jazyce Java, Monkey C kompiluje zdrojové soubory do byte kódu, který je následně interpretován virtuálním strojem Monkey Brains. Virtuální stroj Monkey Brains poté komunikuje s dalšími dostupnými API." \cite{věnsek_2019} Tyto API slouží např. pro práci s polohou zařízení, komunikací se senzory detekující nejrůznější informace (teplotu ovzduší, tlak vzduchu, puls, atd...).


\begin{figure}
	\centering
	\includegraphics{images/code_snippet}
	\\
	\caption{ukázka jednoduchého fragmentu kódu v MonkeyC}
	\label{img:monkeyC_Fragment}
\end{figure}

\subsection{Monkey C a ostatní jazyky}
Stejně jako italština a španělština vznikly z latiny, Monkey C čerpá spostu věcí převážně z jiných moderních jazyků. C, Java, JavaScript, Python, Lua, Ruby a PHP, všechny tyto jazyky ovlivnily výslednou podobu Monkey C. \cite{monkeyc_2021} \\
Mezi Monkey C a výše uvedenýmy jazyky jsou přesto jisté rozdíly, které budou popsány v následujících řádcích.
\begin{enumerate}
\item \textbf{Java} - Stejně, jako Java, je Monkey C kompilování do byte kódu, který je následně interpretován virtuálním strojem. Jak už bylo zmíněno výše, virtuální stroj pro Monkey C nese název Monkey Brains. Další společná vlastnost s Javou je ta, že k alokaci objektů dochází na haldě (heap). V neposledí řadě stojí za zmínku, že virtuální stroj má na starosti čištění paměti, přičemž v Javě je toto realizováno prostřednictvím tzv. "garbage collectoru" a v Monkey C pomocí "reference countingu".
\item \textbf{JavaScript} - Hlavním rozdílem mezi mezi JavaScriptem a Monkey C spočívá v tom, že funkce v Money C nedisponují vlastností "funkce první třídy" (first-class function). "To znamená, že jazyk podporuje předávání funkcí jako argumentů jiným funkcím, jejich vrácení jako hodnoty z jiných funkcí a jejich přiřazení k proměnným nebo jejich uložení v datových strukturách." \cite{abelson1996structure}\\ Na obrázku \ref{img:callback_error} lze vidět, že při pokusu předat funkci, jako parametr jinou funkci, rozšíření detekuje operaci, která je v rozporu s Monkey C gramatikou, a tím pádem oznámí uživateli chybu.
\item \textbf{Python} - Objekty v Pythonu jsou reprezentovány, jako hashovací tabulky, přičemž funkce a proměnné lze objektům přiřazovat za běhu programu. Objekty v Monkey C jsou kompilovány ještě před spuštěním programu, a tím pádem nemohou být modifikovány za běhu. Všechny proměnné, předtím než je použijeme např. ve funkci, třídní instanci či rodičovském modulu, musí být deklarovány.
\end{enumerate}

\begin{figure}[h!]
	\centering
	\includegraphics{images/callback_error}
	\\
	\caption{rozšíření při pokusu předat funkci, jako parametr, detekuje chybu.}
	\label{img:callback_error}
\end{figure}

\endinput
\chapter{Problematika vývoje rozšíření pro VS Code}

Jak už bylo zmíněno v úvodní kapitole, na oficiálním webu s rozšířeními pro VS Code \cite{marketplace} najdeme spousty aplikací, které vývojářům pomáhají např. s "debuggováním" kódu, analýzou dat, strojovým učením (machine learning), atd... Rozšíření pro Monkey C však nejsou zastoupena v tak hojném počtu, jako ostatní jazyky, což lze vidět v tabulce níže \ref{tab:statsTable}. V této kapitole budou popsány komponenty, které jsou stěžejní pro vytvoření překladače jazyka, parseru jazyka a následně rozšíření samotného.\\

\begin{table}
	\centering
	\caption{počet výsledků po vyhledání daného jazyka na marketplace \cite{marketplace}}
	\label{tab:statsTable}
	\begin{tabular}{lr}
			\toprule
			název jazyka & počet výsledků\\
			\midrule
			C Sharp & 169 \\
			Java & 2521 \\
			Python & 454 \\
			R & 6350 \\
			Monkey C & 2 \\
			\bottomrule
	\end{tabular}
\end{table}

\section{Visual Studio Code}
Visual Studio Code \cite{VSCODE_2020} je editor zdrojového kódu vytvořený společností Microsoft pro operační Windows, Linux a MacOS, který podporuje stovky druhů jazyků. Je naprogramován v jazycích JavaScript a Typescript a nabízí spoustu užitečných funkcí, mezi které patří podpora ladění kodu, zvýrazňování syntaxe, automatické doplňování a našeptávání kódu, odsazování textu, intuitivní klávesové zkratky, které usnadňují navigaci v kódu, atd… Cílem práce je integrovat většinu těchto funkcí a možností do finálního rozšíření.

\section{Typescript}
TypeScript \cite{TypeScript_wikipedia_2020} je open-source programovací jazyk vyvinutý společností Microsoft. Jedná se o nádstavbu nad jazykem JavaScript,
která jej rozšiřuje o statické typování a další atributy, které známe z objektově orientovaného programování jako jsou třídy, moduly a další. Samotný kód psaný v jazyce TypeScript se kompiluje do jazyka JavaScript. Jelikož je tento jazyk nádstavbou nad JavaScriptem, je každý JavaScript kód automaticky validním TypeScript kódem.\\
S programovacím jazykem Typescript jsem na začátku této práce mnohem menší zkušenosti, než např. s programovacími jazyky C Sharp nebo JavaScript. V průběhu studia jsem se však setkal s JavaScriptem při vývoji internetových aplikací. A jelikož je Typescript pouze nádstavba JavaScriptu a objektově orientované programování mě bylo dobře známé z jazyka C Sharp, nebylo obtížné se veškeré chybějící potřebné znalosti rychle doučit.

\section{Komponenty potřebné pro tvorbu rozšíření}
Předtím, než bude možné začít pracovat na samotném rozšíření, je potřeba obstarat několik důležitých nástrojů a komponent, jenž jsou klíčové při vývoji. Nejdůležitějším nástrojem pro vývoj rozšíření je však ANTLR, který je popsán v sekci \ref{antlr}.

\subsection{Gramatika}
Jako první je potřeba definovat popis jazyka Monkey C. V tomto případě je jazyk Monkey C popsán prostřednictvím bezkontextové gramatiky, která formálně definuje syntax (pravidla) jazyka. Jedná se, ve své podstatě, o soubor pravidel, kde každé pravidlo reprezentuje určitou strukturu či frázi jazyka. Z tohoto formálního popisu je ANTLR schopen vygenerovat parser daného jazyka, který dokáže automaticky sestavit datovou strukturu, která se nazývá buďto "parse tree" nebo "syntax tree", což v češtině znamená \textbf{syntaktický strom}. Tento syntaktický strom poté reprezentuje, jak přesně je gramatika schopna rozpoznávat vstupní data (např. fragment kódu).\\
Jelikož je v této práci použit ANTLR verze 4, je klíčové, aby název souboru obsahující gramatiku končil příponou .g4, v tomto případě soubor nese název \textbf{MonkeyC.g4}. 
\\
Na obrázku \ref{img:grammar} lze vidět prvních pár řádků Monkey C gramatiky. Gramatika obsahuje spoustu známých klíčových slov, nebo-li tokenů, např. "CLASS", "FUNCTION", "USING", atd...

\section{ANTLR - Nástroj pro generování překladače} \label{antlr}
ANTLR nebo také ANother Tool for Language Recognition (jiný nástroj pro rozpoznávání jazyka) \cite{ANTLR_2021} je výkonný nástroj pro generování syntaktických analyzátorů, tzv. "parserů". Tento parser je poté schopen číst, zpracovávat, spouštět nebo překládat strukturované textové či binární soubory. Používá se především k vytváření nových jazyků, nástrojů nebo "frameworků". Využívá bezkontextový jazyk typu LL, což je syntaktický analyzátor typu "shora-dolů" pro bezkontextové gramatiky. Analyzuje vstup zleva (Left) doprava a konstruuje nejlevější derivaci (Leftmost) věty. Gramatiky, které jsou takto analyzovatelné, se nazývají LL gramatiky.\\
Pro vygenerování Monkey C parseru je tedy potřeba nainstalovat ANTLRv4 (ANTLR verze 4), který je možné získat na oficiálním webu \cite{ANTLR_2021}. Stačí tedy stáhnout aktuální ANTLR jar, což je momentálně "antlr-4.8-complete.jar". Soubor je zakončen příponou \textbf{.jar}, z toho vyplývá, že je ANTLR napsán jazyce Java. K úspěšnému spuštění ANTLR nástroje je vyžadována verze Javy 1.6 a vyšší.

\begin{figure}[b!]
	\centering
	\includegraphics[scale=0.8]{images/grammar}
	\caption{ukázka hlavičky gramatiky MonkeyC.g4 pro popis jazyka}
	\label{img:grammar}
\end{figure}

\begin{figure}
	\centering
	\includegraphics{images/generated_files}
	\\
	\caption{potřebné soubory vygenerované nástrojem ANTLR}
	\label{img:generated_files}
\end{figure}

\subsection{TestRig}
ANTLR poskytuje flexibilní testovací nástroj umístněný v runtime knihovně s názvem TestRig. TestRig dokáže poskytnout spoustu informací o tom, jak "recognizéry" (parser a lexer) zpracovávají InputStream ze vstupního souboru. TestRig je spouštěn z příkazového řádku pomocí aliasu \emph{grun}. Nabízí spoustu možností, jak zobrazit vygenerovaný syntaktický strom:
\begin{enumerate}
\item \textbf{-tree}: zobrazí syntaktický strom jako soubor do sebe zanořených pravidel gramatiky
\item \textbf{-tokens}: výstupem jsou tokeny, které parser vygeneroval ze vstupních dat
\item \textbf{-gui}: zobrazí syntaktický strom vizuálně v programu ParseTreeInspector, který je součástí ANTLR. Příklad zobrazení jednoduchého vstupu bez detekovaných chyb lze vidět na obrázku \ref{img:ParseTreeInspector}. Dále lze na obrázku vidět, jak jsou mezi sebou jednotlivé části stromu, tedy uzly, navzájem propojeny. Strom začíná kořenem, jenž je v gramatice pojmenován, jako "program". Takto je kořen pojmenován při každém parsování. 
\end{enumerate}

Podle mého názoru se jedná o užitečný nástroj, který uživateli poskytuje přehled o tom, zda parser rozpoznal vstupní data správně, případně kde parser detekovat rozpory z gramatikou. Při vývoji a testování rozšíření byl tento nástroj velmi kvalitním pomocníkem. Možností, které pro testování gramatik a parserů TestRig nabízí, je samozřejmě více, ale pro účely této práce byly použity hlavně tři výše uvedené. 

\begin{figure}
	\centering
	\includegraphics[scale=0.8]{images/ParseTreeInspector}
	\caption{"ParseTreeInspector" - vizuální podoba syntaktického stromu} 
	\label{img:ParseTreeInspector}
\end{figure}

\endinput
\chapter{Syntaktická a sémantická analýza kódu}
Abychom mohli implementovat náš jazyk (MonkeyC), je potřeba vytvořit aplikaci, která je schopna číst "věty" na vstupu a odpovídajicím způsobem rozpoznává klíčová slova či fráze naší gramatiky.(Obecně je jazyk složení platných vět, kdy věta se skládá z frází a fráze se skládá ze symbolů slovní zásoby).\\
Obecně lze říci, pokud nějaká aplikace vykonává (interpretuje) zápis jiného programu v jeho zdrojovém kódu ve zvoleném programovacím jazyce (v našem případě Typescript), že se jedná o tzv. interpret \cite{Interpret_2020}. Jako příklady uvedu kalkulačku, aplikace pro čtení konfiguračních souborů, Python interprety, atd... Pokud, na druhou stranu, převádíme "věty" z jednoho jazyka do druhého (např. z Javy do 'C Sharp'), nazýváme takovou aplikaci překladačem.\\
Úkolem překladače či interpreta je tedy rozpoznat platné vstupy, tedy věty, fráze, subfráze, klíčová slova, atd... Přičemž rozpoznáním platného vstupu je myšleno identifikovat jednotlivé fráze a rozližit je od jiných.\\
Jako příklad použijeme rozpoznání vstupu $input = 1;$ s validní MonkeyC syntaxí.  Víme tedy, že "input" je cíl přiřazení a "1" představuje hodnotum která se má přiřadit. Stejně jako jsme my schopni rozližit v našem jazyce sloveso od podstatného jména, je naše aplikace schopna rozlišit toto přiřazení od např. importu knihovny pomocí klíčového slova "using".\\
Programy, které jsou schopny rozpoznávat konkrétní jazyk, se nazývají parsery nebo syntaktické analyzátory, přičemž syntax odkazuje na pravidla obsažená v popisu jazyka, čili gramatice.\\

\begin{figure}
	\centering
	\includegraphics[scale=1.5]{images/assigment}
	 \caption{přiřazení hodnoty 1 proměnné input}
	 \label{img:assigment}
\end{figure}


\section{ANTLR - Nástroj pro generování překladače}
ANTLR nebo také ANother Tool for Language Recognition (jiný nástroj pro rozpoznávání jazyka) \cite{ANTLR_2021} je výkonný nástroj pro generování syntaktických analyzátorů, tzv. "parser". Tento parser je poté schopen číst, zpracovávat, spouštět nebo překládat strukturované textové či binární soubory. Použivá se především k vytváření nových jazyků, nástrojů nebo "frameworků". Využívá bezkontextový jazyk typu LL.
\subsection{LL syntaktický analyzátor}
LL \cite{LL_2017} je syntaktický analyzátor shora-dolů pro bezkontextové gramatiky. Analyzuje vstup zleva (Left) doprava a konstruuje nejlevější derivaci (Leftmost) věty. Gramatiky, které jsou takto analyzovatelné, se nazývají LL gramatiky.

\section{Syntaktický strom}
Syntaktický strom představuje datovou strukturu, kterou ANTLR vytvoří při parsování vstupního souboru. Struktura je složená z kořene (root) a uzlů, které mohou představovat buď další podstromy, které odpovídají pravidlům gramatiky, nebo listy stromu.\\ 
V naší aplikaci Syntaktický strom představuje třída AST (viz. kód níže), která uchovává všechna důležitá data pro korektní analýzu vstupního souboru, např. počet uzlů stromu, soubor vztahující se ke konkrétnímu stromu, kořen stromu, atd...

\begin{figure}
	\centering
	\includegraphics[scale=0.7]{images/parser}
	\caption{Ukázka, jak Language recognizer zpracovává vstupní sekvenci znaků} 			    \cite{ANTLR_PG_10}
	\label{img:parser}
\end{figure}

\lstinputlisting[label=code:AST_TS,caption={třída AST v jazyce Typescript}]{SourceCodes/AST.ts}

Další, neméně důležitou, třídou v naší apikaci je třída Node, která představuje uzel stromu. Jedná se, ve své podstatě, o strukturu uchovávající následující atributy:\\
\begin{enumerate}
	\item kontext daného uzlu
	\item předka uzlu
	\item potomka uzlu
	\item hodnotu přestavující text v daném uzlu
	\item typ uzlu, který se odvíjí od pravidel v gramatice
\end{enumerate}

\lstinputlisting[label=code:Node_TS,caption={třída Node v jazyce Typescript}]{SourceCodes/Node.ts}


\section{ANTLR Listener a callback funkce}
ANTLR disponuje dvěma mechanismy, které umožňují průchod stromem (tzv. "tree-walking mechanisms"). Jedná se o mechanismy \textbf{listener} a \textbf{visitor},pričemž výchozím je listener. Největší rozdíl mezi nimi spočívá v tom, že listener metody jsou volány nezávisle ANTLR objektem, zatímco visitor metody vyvolávají rovněž metody potomků uzlu, na kterém se právě nachází, což způsobuje, že některé podstomy nebudou při průchodu vůbec navštíveny. Samotné listenery jsou ekvivalentí SAX objektům, které se používají v XML parserech.\\
Aby bylo možné stromem procházet a při průchodu vyvolat odpovídající události, ANTLR dispinuje třídou ParseTreeWalker. Právě ona se stará o to, aby byly volány callback funkce. Další důležitou komponentou, vygenerovanou ANTLR nástrojem je rozhraní ParseTreeListener (v našem případě MonkeyCListener \ref{img:generated_files}). Toto rozhraní definuje veškeré metody potřebné k kompletnímu průchodu stromem, např. "enterEveryRule(context: ParserRuleContext)" či "exitStatement(context: StatementContext)". Ve chvíli kdy ParseTreeWalker narazí na příslušný uzel, vyvolá odpovídající metodu. TODO: příklad kdy narazí na nějaké assign třeba


\section{Sémantický strom}
Strom, jenž hraje zásadní roli v procesech, jako je našeptávání kódu, detekci chyb, kdy je použita třída z modulu, který není referencován, atd...

\section{parser}
Fusce nibh. Sed ut perspiciatis unde omnis iste natus error sit voluptatem accusantium doloremque laudantium, totam rem aperiam, eaque ipsa quae ab illo inventore veritatis et quasi architecto 
beatae vitae dicta sunt explicabo. Quis autem vel eum iure reprehenderit qui in ea voluptate velit esse quam nihil molestiae consequatur, vel illum qui dolorem eum fugiat quo voluptas nulla pariatur? Etiam ligula pede, sagittis quis, interdum ultricies, scelerisque eu. Maecenas sollicitudin. Cras pede libero, dapibus nec, pretium sit amet, tempor quis. Integer vulputate sem a nibh rutrum consequat. Pellentesque sapien. Pellentesque arcu. Suspendisse nisl. Fusce consectetuer risus a nunc. Etiam dui sem, fermentum vitae, sagittis id, malesuada in, quam. Cum sociis natoque penatibus et magnis dis parturient montes, nascetur ridiculus mus. Nam quis nulla. Nulla non lectus sed nisl molestie malesuada. Duis viverra diam non justo. Sed ac dolor sit amet purus malesuada congue. Aenean id metus id velit ullamcorper pulvinar. Aliquam ornare wisi eu metus. Neque porro quisquam est, qui dolorem ipsum quia dolor sit amet, consectetur, adipisci velit, sed quia non numquam eius modi tempora incidunt ut labore et dolore magnam aliquam quaerat voluptatem.

\endinput

\chapter{Návrh a implementace rozšíření}
Rozšíření bude, jak již bylo nastíněno výše, implementováno v jazyce Typescript. 
K vytvoření rozšíření samotného je potřeba Node.js a Git. Poté je vyžadována instalace programu "Yeoman" "VS Code Extension Generator".

\section{Error Listener}
Slouží k zachytávní syntaktických chyb v kódu, čili chyb, které jsou v rozporu z pravidly gramatiky. Rozšíření chyby vypisuje do konzole, jak jsme ve VS Code zvyklí. Součástí zprávy jsou:
\begin{enumerate}
\item řádek, na kterém se chyba nachází
\item pozice na řádku
\item popis chyby
\end{enumerate}

\section{Modul Toybox}
Tento modul (modul = namespace) obsahuje všechny potřebné třídy, se kterými v Monkey C můžeme pracovat.

\section{Zvýraznění kódu}

Zvýraznění, nebo-li obarvení klíčových slov kódu je realizováno pomocí JSON souboru, jenž pomocí regulárních výrazů v textu vyhledává příslušná slova a ty poté obarvuje.
\\
\begin{figure}
	\centering
	\includegraphics[]{images/uncolored_code}
	\caption{ukázka neobarveného kódu v MonkeyC}
	\label{img:uncolored_code}
\end{figure}

\begin{figure}
	\centering
	\includegraphics[]{images/colored_code}
	\caption{ukázka obarvení kódu v MonkeyC} 
	\label{img:colored_code}
\end{figure}
	
Hned na první pohled je zřejmé, že obarvení poskytuje uživateli větší přehled a orientaci v kódu, jak je vidět na obrázku \ref{img:colored_code}


\section{Automatické doplňování a našeptávání kódu}

Hlavním aktérem při našeptávání vhodných částí kódu je "antlr4-c3 The ANTLR4 Code Completion Core".

Jedná se o "stroj" sloužící k dokončování gramatického agnostického kódu pro analyzátory založené na ANTLR4. Engine c3 je schopen poskytnout kandidáty na doplnění kódu, kteří jsou užiteční pro editory s analyzátory generovanými ANTLR, nezávisle na skutečném jazyku / gramatice použité pro generování.

Původní implementace je poskytována jako "node module" a je psána v jazyce Typescript.

Pro zobrazení možných symbolů ve zdrojovém kódu zjevně potřebujete zdroj pro všechny dostupné symboly na dané pozici. Jejich poskytnutí je obvykle úkolem tabulky symbolů. Jeho obsah lze odvodit z vašeho aktuálního zdrojového kódu (pomocí analyzátoru + analyzátoru Listeneru). Statičtější části (například runtime funkce) lze načíst z disku nebo poskytnout pevně zakódovaný seznam atd. Tabulka symbolů pak může odpovědět na vaši otázku ohledně všech symbolů daného typu, které jsou viditelné z dané pozice. Pozice obvykle odpovídá konkrétnímu symbolu v tabulce symbolů a struktura pak umožňuje snadno získat viditelné symboly. Engine c3 je dodáván s malou implementací tabulky symbolů, která však není pro použití knihovny povinná, ale poskytuje snadný start, pokud již nemáte vlastní třídu tabulek symbolů.

Zatímco tabulka symbolů poskytuje symboly daného typu, musíme zjistit, který typ je ve skutečnosti vyžadován. To je úkol enginu c3. V nejjednodušším nastavení vrátí pouze klíčová slova (a další symboly lexerů), která jsou povolena gramatikou pro danou pozici (což je samozřejmě stejná pozice, která se používá k nalezení kontextu pro vyhledávání symbolů ve vaší tabulce symbolů). Klíčová slova jsou pevná sada slov (nebo sekvencí slov), která se obvykle nenachází v tabulce symbolů. Skutečné textové řetězce můžete získat přímo ze slovníku analyzátoru. C3 za ně vrací pouze lexerové tokeny.


\begin{figure}[h!]
	\centering
	\includegraphics{images/autocomplete_example}
	\caption{ukázka automatického našeptávání kódu na proměnné typy string}
	\label{img:autocomplete_example}
\end{figure}

\section{komentáře}
Základním prvkem pro práci s komentáři je třída ToyBox. Jedná se o stěžejní modul (Modul v Monkey C je ekvivalentem Namespace např v. C Sharp), ve kterém jsou obsaženy všechny třídy, funkce, proměnné, se kterými MonkeyC pracuje. Toybox, jako takový není nikde oficiálně dostupný, tudíž bylo nutné tento hlavní modul vygenerovat. Zdrojem pro generování bylo Connect IQ SDK.

\begin{figure}
	\centering
	\includegraphics{images/comments}
	\caption{komentář nad funkcí obsahující stručný popis, parametry funkce a návratový typ}
	\label{img:comments}
\end{figure}

\section{nedostatky rozšíření}
Při implementaci automatického našeptávání byl detekován problém, kvůli kterému není možné provést volání více funkcí po sobě na jednom řádku. Uveďme si příklad, kdy máme proměnnou typu \textbf{String}, v níž je uloženo číslo. Hodnotu v této proměnné budeme chtít převést na typ Integer, čili zavolat metodu \textit{toNumber()}, a poté bezprostředně po volání \textit{toNumber()} zavolat další metodu. Zde nastává problém, kdy rozšíření, jako další vstup neočekává možné volání funkce \ref{img:autocomplete_errormessage}. Jádrem tohoto problému je, že poskytnutá bezkontextová gramatika popisující jazyk není stoprocentně přesná. A právě kvůli těmto "nepřesnostem" je možné při implementaci narazit na podobné komplikace. V průběhu vývoje zatím nebyly registrovány další problémy způsobené gramatikou.

\begin{figure}
	\centering
	\includegraphics[scale=1]{images/autocomplete_error}
	\caption{nedostatek rozšíření}
	\label{img:autocomplete_error}
\end{figure}

\begin{figure}
	\centering
	\includegraphics[scale=0.8]{images/autocomplete_errormessage}
	\caption{chybová hláška z error listeneru}
	\label{img:autocomplete_errormessage}
\end{figure}
\endinput
\chapter{Testování výsledného řešení}
Testování probíhalo pravidelně v průběhu vývoje rozšíření na jednoduchých fragmentech kódu Monkey C jazyka. Pro finální testovaní však byly použity programy, jenž byly součástí reálných aplikací. Tyto příkladové aplikace byly převzaty z oficiální Connect IQ SKD.\\

První testovací zdrojový kód je možné vidět na výpisu \textbf{\ref{testSrc1:Sensor}}. Tento kód komunikuje se senzory obsažených v Garmin zařízení a s jejich pomocí měří srdeční tep uživatele. Při testování tohoto kódu nedošlo k žádným výraznějším problémům. Při deklaraci jednotlivých tříd, které vždy byly rozšířeny jinou třídou pomocí \textbf{extends}, rozšíření správně našeptávalo možné kandidáty na doplnění. Dále ve funkci \textit{initialize()}, kde je volána metoda \textit{enableSensorEvents()}, bylo rozšíření schopno našeptávat callback funkce po zadání znaku '\textbf{:}'. Některé části kódu bylo však nutné upravit. Jako první bylo nutné přidat komentáře s datovým typem nad deklarace proměnných\textit{ stringHR} a \textit{HRgraph}. Zde však nastal problém, kdy nebylo možné jednoznačně určit, jaký datový typ má proměnná \textit{HRgraph} obsahovat. Ve funkci \textit{initialize()} je do této proměnné vložena instance třídy \textbf{LineGraph}. Tuto třídu však rozšíření nenašlo a po zkontrolování oficiální Monkey C API \cite{api_docs} dokumentace bylo zjištěno, že takovou třídu API neobsahuje, tudíž ji rozšíření ani nemohlo nabídnout, jako kandidáta na doplnění.
\\
Druhý testovací zdrojový kód je možné vidět na výpisu \textbf{\ref{testSrc2:BurstChnl}}. Jedná se o část aplikace GenericChannelBurst, která pracuje s třídami Toybox.Ant.BurstListener, Toybox.Ant.Ant.GenericChannel,  Toybox.Ant.ChannelAssignment, atd... U toho zdrojového kódu byl kladen důraz především na práci s konstantami. Hned před deklarací třídy \textit{BurstChannel} lze vidět první konstantu \textbf{ANT-DATA-PACKET-SIZE}. Další konstanty následují již v těle třídy. Na obrázku \ref{img:constants} lze vidět, jak rozšíření našeptává konstanty obsahující písmeno \textbf{D}. Lze si také všimnout, že i lokální proměnné deklarované uživatelem obsahují vlastní popis. V tomto případě je součástí popisu datový typ proměnné. Při importovaní modulů pomocí \textbf{using}, deklaraci třídy společně s použití dědičnosti a deklaraci všech proměnných nebyl detekován žádný problém. Dále byla testována deklarace nového pole, práce s \textbf{for} cyklem, volaní metod na proměnných atd... 

\begin{figure}[tbh!]
	\centering
	\includegraphics[width=\textwidth,scale=1]{images/constants}
	\caption{rozšíření našeptává konstanty obsahující písmeno D.}
	\label{img:constants}
\end{figure}
\chapter{Závěr}
Cílem této bakalářské práce bylo nastudovat jazyk Monkey C, nástroj ANTLR, díky kterému jsme schopni analyzovat a parsovat formální jazyky, a s jeho pomocí vytvořit parser jazyka Monkey C, jenž byl následně použit k analyzování kódu. 
\\
V úvodní části této práce se hovoří o tom, jak znatelně malé zastoupení má jazyk Monkey C na trhu s rozšířeními pro VS Code, vzhledem k ostatním programovacím jazykům. Tato skutečnost byla také jeden z hlavních důvodů, které vedly k tvorbě vlastního řešení. Dále byl v kapitole \ref{2Chapter} představen a popsán jazyk Monkey C, který je určen k vývoji aplikací a rozšíření pro zařízení Garmin, byly posány druhy aplikací, které je možné v Monkey C vyvíjet. Závěrem této kapitoly bylo porovnání Monkey C s ostatními moderními programovacími jazyky. Při vývoji rozšíření a následném testování jsem došel k závěru, že jazyk Monkey C má s ostatními, jako Python či Java mnoho společného. 
\\
Hlavním výsledkem práce je rozšíření pro VS Code, které uživateli napomáhá s psaním Monkey C kódu. Návrhem a implementací se zabývala kapitola \ref{5Chapter}. Byly zde popsány všechny důležité komponenty, jako providery pro našeptávání kódu, třída \textit{extension.ts}, ve které jsou providery implementovány, dále třída \textit{documentHandler.ts}, která obsahuje Mapy pro sémantickou analýzu a další podpůrné funkce pro parsování dokumentů, komunikaci s třídou \textit{Listener.ts}, která obsahuje callback funkce,e jenž odpovídají pravidlům gramatiky popsané v kapitole \ref{4Chapter}, atd... Během vývoje byly řešeny problému, jako jsou detekce modulů, funkcí, tříd a proměnných. Bylo řešeno našeptávání proměnných a funkcí, které jsou obsaženy ve třídě, pomocí klíčových slov \textbf{self.} a \textbf{me.}, které Monkey C využívá. Bylo řešeno parsování všech souborů, pokud se jich v pracovní složce nachází více. Dále byla řešena viditelnost mezi soubory, obarvení kódu pro jeho zpřehlednění, výpis chyb prostřednictvím ErrorListeneru, detekování datového typu proměnné pomocí komentáře, jenž je její součástí, atd... Součástí práce je i vygenerovaný modul Toybox, obsahující všechny potřebné třídy a funkce. Rozšíření bylo testováno na ukázkových kódech z oficiálního Connect IQ SDK.
\\
V této práci jsem uplatnil mnoho znalostí napříč různými jazyky. Využil jsem znalost jazyka JavaScript, jenž má prakticky totožnou syntax s Typescriptem. Dále jsem využil znalosti objektově orientovaného programování, které byly velmi užitečné při vývoji všech podpůrných tříd použitých ve finální aplikaci.\\
Díky této práci jsem měl také možnost rozšířit své znalosti o práci s jazykem Typescript. Dále jsem měl možnost pracovat s nástrojem ANTLR, který mě umožnil nahlédnout do problematiky překladačů a vývoji jazykových procesorů. Po dobu tvorby práce byl veškerý postup zaznamenáván pomocí verzovacího nástroje Git.
\\
Do budoucna by bylo možné třídu DocumentHandler rozšířit o tabulku symbolů, která by více zefektivnila sémantickou analýzu kódu, a tím pádem zvýšila účinnost a rychlost rozšíření. Dále by bylo vhodné nalézt efektivnější řešení pro generovaní Toybox modulu, jenž byl pro tuto práci vygenerován z html elementů obsažených v SDK. Jedním z možných řešení by bylo, kdyby společnost Garmin Ltd. \cite{GARMIN_OFFICIAL} tento modul v nějaké ucelené podobě zveřejnila.

\endinput

% Seznam literatury
\printbibliography[title={Literatura}, heading=bibintoc]

% Prilohy
\appendix
% Priloha vlozena primo do hlavniho LaTeX souboru. Ne vsechny prilohy je nutne mit ve zvlastnich souborech.
\chapter{Testovací zdrojové kódy}
\lstinputlisting[label=testSrc1:Sensor,caption={Testovací zdrojový kód 1}]{SourceCodes/Sensor.mc}

\lstinputlisting[label=testSrc2:BurstChnl,caption={Testovací zdrojový kód 2}]{SourceCodes/BurstChannelManager.mc}

\end{document}

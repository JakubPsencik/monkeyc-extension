\chapter{Úvod}
\label{sec:Introduction}
V dnešní době, kdy jsme obklopeni spoustou moderních technologií, existující programovací jazyky se stále vyvíjejí kupředu a nové postupně vznikají, existuje nespočet nástrojů, rozšíření, vývojových prostředí, které práci a vývoj v těchto jazycích dokážou v mnoha ohledech usnadnit. Jazyk Monkey C, jenž bude středobodem této práce, zatím nedisponuje tak široce obsáhlou komunitou, jako mají např. v dněšní době velmi populární Javascript, Python, C Sharp, Ruby, atd...\\

Typickým příkladem společnosti, která se zabývá právě vývojem softwarů pro programátory či vývojáře, je česká JetBrains s.r.o. Ti mají na kontě již mnoho produktů usnadňujících programování, např. ReSharper (.NET), IntelliJ IDEA (Java, Groovy, atd...) či PyCharm (Python). Ovšem na podporu Monkey C zatím žádné softwarové řešení v JetBrains do světa nevpustili. \cite{jetbrains} \\

Na oficiálním webu, kde Microsoft nabízí nejrůznější rozšíření  \cite{marketplace}, je sice možné nalézt několik nástrojů, které s vývojem v Monkey C pomáhají. Cílem této práce je však tvorba rozšíření, které poskytne uživateli co největší podporu a hlavně, aby byla postavena na vlastním řešení.
\\
Následujících kapitoly tedy budou věnovány tvorbě a vývoji rozšíření pro vývojové prostředí Visual Studio Code, které poskytne plnou podporu při vývoji aplikací v Monkey C. V kapitole 2 dojde k představení samotného jazyka Monkey C a operačního systému Connect IQ, na kterém Monkey C běží. V kapitole 3 jsou popsány veškeré nástroje a komponenty, které budou při tvorbě rozšíření použity. Pro tvorbu rozšíření bude využit jazyk Typescript, dále pak nástroj ANTLR, který je schopen vygenerovat překladač jazyka. Toho je schopen dosáhnout za pomoci jeho popisu obsaženého v bezkontextové gramatice, jenž byla pro tuto práci poskytnuta, a tudíž její vytváření nebylo součástí práce. Kapitola 4 se popisuje analýzu kódu jak z pohledu syntaxe, tak sémantiky. V kapitole 5 se práce zabývá již samotným návrhem a implementací rozšíření. Objeví se zde témata, jako parsování kódu, automatické doplňování a našeptávání kódu, atd... Kapitola 6 je poté věnována testování rozšíření.
 
\endinput
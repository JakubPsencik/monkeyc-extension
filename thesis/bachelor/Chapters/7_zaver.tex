\chapter{Závěr}
Cílem této bakalářské práce bylo nastudovat jazyk Monkey C, nástroj ANTLR, díky kterému jsme schopni analyzovat a parsovat formální jazyky, a s jeho pomocí vytvořit parser jazyka Monkey C, jenž byl následně použit k analyzování kódu. 
\\
V úvodní části této práce se hovoří o tom, jak znatelně malé zastoupení má jazyk Monkey C na trhu s rozšířeními pro VS Code, vzhledem k ostatním programovacím jazykům. Tato skutečnost byla také jeden z hlavních důvodů, které vedly k vytvoření vlastního řešení. Dále byl popsán jazyk Monkey C, který je určen k vývoji aplikací a rozšíření pro zařízení Garmin, a druhy aplikací, které je možné v Monkey C vyvíjet. Při vývoji rozšíření a následném testování jsem došel k závěru, že jazyk Monkey C má s ostatními, jako Python či Java mnoho společného. 
\\
Hlavním výsledkem práce je rozšíření pro VS Code, které uživateli napomáhá s psaním Monkey C kódu. Během vývoje byly řešeny problému, jako jsou detekce modulů, funkcí, tříd a proměnných. Bylo řešeno našeptávání proměnných a funkcí, které jsou obsaženy ve třídě, pomocí klíčových slov \textbf{self.} a \textbf{me.}, které Monkey C využívá. Bylo řešeno parsování více souborů, pokud se jich v pracovní složce nachází více. Dále byla řešena viditelnost mezi soubory, obarvení kódu pro jeho zpřehlednění, výpis chyb prostřednictvím ErrorListeneru, detekování datového typu proměnné pomocí komentáře, jenž je její součástí, atd... Součástí práce je i vygenerovaný modul Toybox, obsahující všechny potřebné třídy a funkce. Rozšíření bylo testováno na ukázkových kódem z oficiálního Connect IQ SDK.
\\
V této práci jsem uplatnil mnoho znalostí napříč různými jazyky. Využil jsem znalost jazyka JavaScript, jenž má prakticky totožnou syntax s Typescriptem. Dále jsem využil znalosti objektově orientovaného programování, které byly velmi užitečné při vývoji všech podpůrných tříd použitých ve finální aplikaci.\\
Díky této práci jsem měl také možnost rozšířit své znalosti o práci s jazykem Typescript. Dále jsem měl možnost pracovat s nástrojem ANTLR, který mě umožnil nahlédnout do problematiky překladačů a vývoji jazykových procesorů. Po dobu tvorby práce byl veškerý postup zaznamenáván pomocí verzovacího nástroje Git.
\\
Do budoucna by bylo možné třídu DocumentHandler rozšířit o tabulku symbolů, která by více zefektivnila sémantickou analýzu kódu, a tím pádem zvýšila účinnost rozšíření. Dále by bylo vhodné nalézt efektivnější řešení pro generovaní Toybox modulu, jenž byl pro tuto práci vygenerován z html elementů obsažených v SDK. Jedním z možných řešení by bylo, kdyby společnost Garmin Ltd. \cite{GARMIN_OFFICIAL} tento modul v nějaké ucelené podobě zveřejnila.

\endinput